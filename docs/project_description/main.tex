\documentclass[12pt,a4paper]{article}
\usepackage{times}
\usepackage{hyperref}
\usepackage{wrapfig}
\usepackage{latexsym}
\usepackage{subcaption}
\usepackage{csquotes}
\usepackage[round]{natbib}
\usepackage{multirow}
\usepackage{lipsum}
\usepackage[table]{xcolor}
\usepackage{placeins}
\usepackage{tikz}
\usepackage{amsmath}
\usepackage{graphicx}
\usepackage{geometry}
%\usepackage[hidelinks]{hyperref}
\usepackage{cleveref}
\usepackage{setspace}
\usepackage{ragged2e}
\usepackage{array}
\usepackage[font=footnotesize]{caption}
\usepackage{threeparttable}
\usepackage{threeparttablex}
\usepackage{longtable}
\usepackage{booktabs}
\usepackage{float}
\usepackage{tabulary}
\onehalfspacing
\geometry{top=2.5cm,
		bottom=2.5cm,
        left=3cm,
        right=3cm}
\usepackage{url}
\newcommand\BibTeX{B\textsc{ib}\TeX}

\newcolumntype{L}[1]{>{\RaggedRight\hspace{0pt}}p{#1}}
\newcolumntype{R}[1]{>{\RaggedLeft\hspace{0pt}}p{#1}}
\captionsetup[figure]{font=footnotesize,labelfont=footnotesize,labelfont={bf},labelformat={default},labelsep=period,name={Fig.}, justification=centering,singlelinecheck=false, width = 15cm}
\captionsetup[table]{font=footnotesize,labelfont=footnotesize,labelfont={bf},labelformat={default},labelsep=period,name={Table},justification=centering, singlelinecheck=false, width = 15cm}
\captionsetup[figure]{labelfont={bf},name={Figure},labelsep=period}
\captionsetup[table]{labelfont={bf},name={Table},labelsep=period}

\title{Mining Sentiments and Arguments in United Nations Security Council (UNSC)
  Speeches \\[5pt]
\large Exploring sentiment and argumentation pipelines in the UNSC political
speech corpus detailed in \citet{schnfeld2019security}}

\author{Atreya Shankar, Juliane Hanel\\
\texttt{\{shankar,hanel\}@uni-potsdam.de} \\
PM: Mining Sentiments and Arguments, WiSe 2019/2020 \\
Prof. Dr. Manfred Stede \\
University of Potsdam}

\date{\today}

\begin{document}

\maketitle
\thispagestyle{empty}

\begin{abstract}\label{abstract}
  The UNSC political speech corpus was released in late 2019 with the publication of
  \citet{schnfeld2019security}. The corpus contains $\sim$65,000
  speeches from $\sim$5,000 security council meetings over the years of 1995-2017. As a
  result, it represents a comprehensive corpus which can be utilized for
  sentiment and argumentation mining. This project will explore the applications
  of various sentiment and argumentation mining pipelines into this newly released dataset 
\end{abstract}

\newpage
\setcounter{page}{1}
\thispagestyle{plain}

\section{Project Description}\label{sec:intro}

Here, we present a preliminary project description with ranked preferences of
techniques that we could use to mine the UNSC political speech corpus. As this
project is exploratory in nature, some of these methods might be changed with
future iterations of research. We will be publishing our work and source code in
a public GitHub repository.\footnote{\url{https://github.com/atreyasha/sentiment-argument-mining}}

\subsection{Sentiment Mining}

In order to conduct sentiment mining into the UNSC corpus, we aim to test
multiple pre-developed sentiment analysis frameworks and compare their results
on the UNSC corpus. Some of the approaches we could test are outlined below.

\subsubsection{Lexicoder Sentiment Dictionary (LSD)}

This is a lexicon\footnote{\url{http://www.lexicoder.com/}} designed to capture
the sentiment of political texts. This could be a good starting point;
definitely more specific than other lexica but maybe rather appropriate for
evaluating politics on national level. Some issues are not discussed at the
international/UN level; such as wages, coalitions and pensions.

\subsubsection{VADER} 
VADER is a rule-based sentiment analysis tool that is specifically attuned to
sentiments expressed in social media \citep{vader}. It might be limited since the tool is
optimized for social media sentiment analysis. We could attempt to improve the
functionality of VADER using the aforementioned LSD.

\subsubsection{Textblob}

This is a library\footnote{\url{https://textblob.readthedocs.io/en/dev/}} built
on top of NLTK. It brings in subjectivity as an interesting feature in addition
to sentiment. Sentiment ranges between -1.0 and 1.0, where -1.0 is the most
negative, 0.0 is neutral and 1.0 is the most positive sentiment value.
Subjectivity is within the range 0.0 to 1.0 where 0.0 is very objective and 1.0
is very subjective. This might be interesting for classifying spontaneous speech, as it might be less subjective.

\clearpage
\subsection{Argumentation Mining}

As per our discussions, it would be good if we could attempt a downstream task
of classifying spontaneous and prepared speech segments. However, in order to
achieve this downstream task, we would need to work on breaking the speeches
down into their respective argumentation structures. Following are two ranked
approaches in terms of decreasing preference.

\subsubsection{Joint Pointer Neural Architecture}

In order to break the UNSC speeches into smaller claims and premises, we
could use a joint pointer architecture proposed by \citet{potash2016heres}.
Although the authors did not publicly release their source code, this
architecture was previously emulated by students from the University of
Potsdam.\footnote{\url{https://github.com/oguzserbetci/argmin2017}} As a result,
the source code for this architecture has been made publicly available.

In order to train this neural network, it would be best to use claims and
premises which are relevant to political speeches. For this, we could use US
election debate corpus\footnote{\url{https://github.com/ElecDeb60To16/Dataset}}
which has been summarized in \citet{haddadan-etal-2019-yes}. This corpus is
already annotated for premises and claims. We could train a joint pointer neural
network and compare the classification results against that of baseline models
in the paper.

Following this, we could extend the application of the trained joint pointer
neural network to the UNSC political speech corpus. As a downstream task,
identifying potential claims/premises would require basic filtering using
discourse connectives, and we could further analyze this aspect.

\subsubsection{Political-Domain Ontology}

The aforementioned joint neural network solution would only provide an
approximate solution to this problem of argumentation mining, particularly due to
lack of a global ontology for the political domain. A lower ontology designed for the political domain would be very helpful in setting up a proper knowledge base which could be much more scalable and interpretable than machine learning solutions.

However, this would be very costly and generally difficult to do. Some papers
have attempted making ontologies, for example in the medical/cancer domain as
per \citet{ontology}, with reasonable results. It would be interesting to
pursue an ontology as it resembles a global solution to knowledge and argument representation. 

\newpage
\bibliographystyle{plainnat}
\bibliography{bibliography}
\nocite{*}
\end{document}