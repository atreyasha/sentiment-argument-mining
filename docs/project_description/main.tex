\documentclass[12pt,a4paper]{article}
\usepackage{times}
%\usepackage{hyperref}
\usepackage{wrapfig}
\usepackage{latexsym}
\usepackage{subcaption}
\usepackage{csquotes}
\usepackage[round]{natbib}
\usepackage{multirow}
\usepackage{lipsum}
\usepackage[table]{xcolor}
\usepackage{placeins}
\usepackage{tikz}
\usepackage{amsmath}
\usepackage{graphicx}
\usepackage{geometry}
\usepackage[hidelinks]{hyperref}
\usepackage{cleveref}
\usepackage{setspace}
\usepackage{ragged2e}
\usepackage{array}
\usepackage[font=footnotesize]{caption}
\usepackage{threeparttable}
\usepackage{threeparttablex}
\usepackage{longtable}
\usepackage{booktabs}
\usepackage{float}
\usepackage{tabulary}
\onehalfspacing
\geometry{top=2.5cm,
		bottom=2.5cm,
        left=3cm,
        right=3cm}
\usepackage{url}
\newcommand\BibTeX{B\textsc{ib}\TeX}

\newcolumntype{L}[1]{>{\RaggedRight\hspace{0pt}}p{#1}}
\newcolumntype{R}[1]{>{\RaggedLeft\hspace{0pt}}p{#1}}
\captionsetup[figure]{font=footnotesize,labelfont=footnotesize,labelfont={bf},labelformat={default},labelsep=period,name={Fig.}, justification=centering,singlelinecheck=false, width = 15cm}
\captionsetup[table]{font=footnotesize,labelfont=footnotesize,labelfont={bf},labelformat={default},labelsep=period,name={Table},justification=centering, singlelinecheck=false, width = 15cm}
\captionsetup[figure]{labelfont={bf},name={Figure},labelsep=period}
\captionsetup[table]{labelfont={bf},name={Table},labelsep=period}

\title{Mining Sentiments and Arguments in United Nations Security Council (UNSC)
  Speeches \\[5pt]
\large Exploring sentiment and argumentation pipelines in the UNSC corpus
developed in \citet{schnfeld2019security}}

\author{Atreya Shankar, Juliane Hanel\\
\texttt{\{shankar,hanel\}@uni-potsdam.de} \\
PM: Mining Sentiments and Arguments, WiSe 2019/2020 \\
Prof. Dr. Manfred Stede \\
University of Potsdam}

\date{\today}

\begin{document}

\maketitle
\thispagestyle{empty}

\begin{abstract}\label{abstract}
  The UNSC political speech corpus was released with the publication of \citet{schnfeld2019security}
  and represents a comprehensive poltical speech corpus; containing $\sim$65,000
  speeches from 4,400 security council meetings over the years of 1995-2017. As a
  result, it represents a comprehensive corpus which can be utilized for
  sentiment and argumentation mining. This project will explore the applications
  of various sentiment and argumentation mining pipelines into this newly released dataset. 
  
\end{abstract}

\newpage
\setcounter{page}{1}
\thispagestyle{plain}

\section{Project Description}\label{sec:intro}

Here, we present a preliminary project description with ranked preferences of
techniques that we could use to mine the UNSC political speech corpus. As this
project is exploratory in nature, some of these methods might be changed with
future iterations of research. We will be publishing our work and source code in
a public GitHub repository.\footnote{\url{https://github.com/atreyasha/sentiment-argument-mining}}

\subsection{Sentiment Mining}

\subsection{Argumentation Mining}

\newpage
\bibliographystyle{plainnat}
\bibliography{bibliography}
\nocite{*}
\end{document}