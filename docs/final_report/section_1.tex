\section{Introduction}

The United Nations Security Council (UNSC) corpus, detailed in \citet{schnfeld2019security}, is a novel political speech corpus containing 65,393 textual speech records from 4,460 security council meetings over the years of 1995-2017. Due to its large size and wide temporal distribution, it could be a very useful corpus for various Natural Language Processing (NLP) tasks in the political domain.

One limitation of the current version of this corpus is its lack of extensive handwritten annotations \citep{schnfeld2019security}; which diminishes its utility for downstream supervised NLP tasks. In order to address this limitation, our project aims to evaluate and provide machine-driven sentiment and argumentation annotations for this corpus. We utilize state-of-the-art tools for this purpose, which include \texttt{VADER}, \texttt{AFINN}, and \texttt{TextBlob} for sentiment analysis and a political-domain fine-tuned version of the \texttt{ALBERT} language model for argumentation mining. Our automatic annotations, while not being as reliable as human annotations, nevertheless provide an initial foothold for future human annotations to follow-through.

In section \ref{background}, we describe and define various background concepts that are used for the methodologies in this paper. In section \ref{methods}, we describe the various methodologies implemented in this study. In sections \ref{results} and \ref{discuss}, we describe our results and discuss their implications. Lastly, we conclude this study in section \ref{conclu} and provide some recommendations for future work in section \ref{reco}.