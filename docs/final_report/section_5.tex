\section{Discussion}
\label{discuss}

\subsection{Sentiment Analysis}
The results of the sentiment analysis seem to reveal a correlation of negative sentiment polarity and objectivity. The trend can be observed best in active countries for which we have a lot of data, e.g. the UNSC core nations and Germany. In order to confirm the validity of this correlation assumption, we would have to find viable means of evaluation. For this project, we manually evaluated a small number of speeches in order to get a grasp of the accuracy of our computed subjectivity and sentiment scores. The evaluation is not a simple task, as we are handling almost 65,000 speeches, and even evaluating 1\% of the data manually would be a labor-intensive tasks.

While performing the manual evaluation, we realized that the sentiment scoring yielded very positive results for rather neutral speeches. We also observed that the subjectivity scores were mostly located around the 0.5 mark, indicating the speech is somewhere in between objective and subjective. We assume that this is due to the fact that the United Nations Security Council speeches use unusual language. The English is very formal and lacks colloquial words and intensifiers. This kind of formal language, combined with the specific defense related topics discussed, might not be captured well by the two sentiment frameworks we used for the analysis. For the subjectivity scores, we came to the conclusion that due to the spokesperson speaking on behalf of an entire country, the typical indicators of subjective speech are not necessarily present, even if the spokesperson expresses their own opinion.

After close inspection of speeches from an extensive set of countries, we also noticed patterns that could have possibly lead to the high subjectivity polarity in some cases. It is part of the rules of etiquette to thank the previous speaker and in some cases elaborate on the positive work they did relating to a topic. At the end of speeches, the spokesperson or a moderator introduces the new person in a similar fashion. These sections of affirmation are repetitive and not related to the topics themselves, yet they are included in the scoring since they are part of the speeches. This skews the scores and raises the sentiment polarity to a higher level.
Furthermore, the speeches are oftentimes translated from the spokesperson's mother tongue to English, which can deprive the speeches of unique facets the speaker wanted to express since they can get lost in translation. 

One last aspect we noticed was that the speeches are stripped of environmental factors, e.g. interruptions that happened or hesitation on the part of the speaker. This would add valuable information to both levels of our analysis.

Despite the issues we noticed during the evaluation, we believe our sentiment analysis provides interesting insights into the UNSC speech data. While the results should be taken with a grain of salt, they still depict the differing sentiments among participants. Especially the insights into the Iraq War related sessions allow for an interesting digression into this controversial topic.

Said insights revealed for example that the opponents of this military intervention had a lower ratio of uttered negative vs. positive sentences than the proponents did. Even after a simple visual exploration of the plots, it is evident that the USA were the driving force in this issue, as they contributed the most to the sessions related to Iraq. Especially in the second session we analyzed, the session that opened with Colin Powell's speech, it can be seen that the two proponents of the invasion of Iraq, the USA and the UK, had a very negative vocabulary. While the UK did not contribute as much to the topic as the USA, more than 50\% of their contributions were negative. This degree of negativity cannot be observed in countries that held other opinions on the Iraq War. 

\subsection{Argumentation Mining}

\subsubsection{Model Performance Comparison}

Since the USED corpus is a relatively new political-domain argumentation corpus, we were unable to find directly comparable results from other studies. However, we can make some approximate comparisons from related studies. As mentioned in \ref{used}, \citet{haddadan-etal-2019-yes} performed argumentation classification of the USED corpus; albeit at a coarser sentence-level instead of the token-level. Their best model achieved 84.3$\%$ F$_1$ score for argumentative sentence identification and 67.3$\%$ F$_1$ score on claim/premise sentence classification. The better results could have stemmed from an intrinsically simpler sentence-level classification task.

\citet{eger2017neural} conducted a similar methodology as ours on the PEC; with their best model achieving a 75.6$\%$ F$_1$ score for the argumentation tagging task. This result is definitely a positive one, however we would question the robustness of such a model due to likely symbolic overfitting on the small training PEC containing only 402 essays.

\subsubsection{Prediction on UNSC}
\label{manual_semantic}

Due to time and resource limitations, we only manually review two predictions from our fine-tuned classifier on the UNSC corpus. We use the same coloring scheme for N, C and P tokens as per Figures \ref{used_distribution_combined} and \ref{unsc_pred}. In the positive example, we can observe clear and expansive segmentation with the claim and premise being in appropriate locations with a discourse connective ``but" between them. In the negative example, we can observe much more fragmentation of token spans; with the true premise after ``because" being (mostly) wrongly labelled as a claim. The fragmentation of token spans is not entirely surprising, since such fragmented argumentation spans also exist in the USED corpus. The true premise being wrongly predicted as a claim is however a limitation of the classifier in this example.

\begin{description}[style=nextline]

\item[Colour Scheme:]

\colorbox{red!60!white}{\strut None (N)}
\colorbox{LightCyan}{\strut Claim (C)}
\colorbox{LightForestGreen}{\strut Premise (P)}

\item[\text{Positive Example: UNSC$\_$2004$\_$SPV.5007$\_$spch019}]
\colorbox{LightForestGreen}{\strut $\_$we} \colorbox{LightForestGreen}{\strut $\_$have} \colorbox{LightForestGreen}{\strut ,} \colorbox{LightForestGreen}{\strut$\_$indeed} \colorbox{LightForestGreen}{\strut ,} \colorbox{LightForestGreen}{\strut$\_$a} \colorbox{LightForestGreen}{\strut$\_$broad} \colorbox{LightForestGreen}{\strut$\_$range} \colorbox{LightForestGreen}{\strut$\_$of} \colorbox{LightForestGreen}{\strut$\_$tools} \colorbox{LightForestGreen}{\strut ,} \colorbox{LightForestGreen}{\strut$\_$developed} \colorbox{LightForestGreen}{\strut$\_$in} \colorbox{LightForestGreen}{\strut$\_$accordance} \colorbox{LightForestGreen}{\strut$\_$with} \colorbox{LightForestGreen}{\strut $\_$chapter} \colorbox{LightForestGreen}{\strut$\_$viii} \colorbox{LightForestGreen}{\strut$\_$of} \colorbox{LightForestGreen}{\strut$\_$the} \colorbox{LightForestGreen}{\strut$\_$charter} \colorbox{LightForestGreen}{\strut ,} \colorbox{LightForestGreen}{\strut$\_$to} \colorbox{LightForestGreen}{\strut$\_$facilitate} \colorbox{LightForestGreen}{\strut$\_$cooperation} \colorbox{LightForestGreen}{\strut .} \colorbox{red!60!white}{\strut$\_$but} \colorbox{LightCyan}{\strut$\_$we} \colorbox{LightCyan}{\strut$\_$need} \colorbox{LightCyan}{\strut$\_$fresh} \colorbox{LightCyan}{\strut$\_$ideas} \colorbox{LightCyan}{\strut$\_$in} \colorbox{LightCyan}{\strut$\_$order} \colorbox{LightCyan}{\strut$\_$to} \colorbox{LightCyan}{\strut$\_$improve} \colorbox{LightCyan}{\strut$\_$such} \colorbox{LightCyan}{\strut$\_$cooperation} \colorbox{LightCyan}{\strut$\_$and} \colorbox{LightCyan}{\strut$\_$to} \colorbox{LightCyan}{\strut$\_$make} \colorbox{LightCyan}{\strut$\_$sure} \colorbox{LightCyan}{\strut$\_$that} \colorbox{LightCyan}{\strut$\_$stability} \colorbox{LightCyan}{\strut$\_$can} \colorbox{LightCyan}{\strut$\_$be} \colorbox{LightCyan}{\strut$\_$achieved} \colorbox{LightCyan}{\strut$\_$as} \colorbox{LightCyan}{\strut$\_$a} \colorbox{LightCyan}{\strut$\_$result} \colorbox{LightCyan}{\strut$\_$of} \colorbox{LightCyan}{\strut$\_$cooperation} \colorbox{LightCyan}{\strut$\_$and} \colorbox{LightCyan}{\strut$\_$interaction} \colorbox{LightCyan}{\strut .}

\item [\text{Negative Example: UNSC$\_$2009$\_$SPV.6075$\_$spch042}]

 \colorbox{red!60!white}{\strut$\_$just} \colorbox{red!60!white}{\strut$\_$very} \colorbox{red!60!white}{\strut$\_$briefly} \colorbox{red!60!white}{\strut,} \colorbox{red!60!white}{\strut$\_$i} \colorbox{LightCyan}{\strut$\_$think} \colorbox{red!60!white}{\strut$\_$i} \colorbox{LightCyan}{\strut$\_$can} \colorbox{LightCyan}{\strut$\_$only} \colorbox{red!60!white}{\strut$\_$endorse} \colorbox{red!60!white}{\strut$\_$what} \colorbox{red!60!white}{\strut$\_$alain} \colorbox{red!60!white}{\strut$\_$le} \colorbox{red!60!white}{\strut$\_$roy} \colorbox{red!60!white}{\strut$\_$has} \colorbox{red!60!white}{\strut$\_$just} \colorbox{red!60!white}{\strut$\_$said} \colorbox{LightCyan}{\strut.} \colorbox{LightCyan}{\strut$\_$we} \colorbox{LightCyan}{\strut$\_$must} \colorbox{LightCyan}{\strut$\_$make} \colorbox{LightCyan}{\strut$\_$sure} \colorbox{LightCyan}{\strut$\_$that} \colorbox{LightCyan}{\strut$\_$we} \colorbox{LightCyan}{\strut$\_$commit} \colorbox{LightCyan}{\strut$\_$ourselves} \colorbox{LightCyan}{\strut$\_$fully} \colorbox{LightCyan}{\strut$\_$to} \colorbox{LightCyan}{\strut$\_$actively} \colorbox{LightCyan}{\strut$\_$participating} \colorbox{LightCyan}{\strut$\_$in} \colorbox{LightCyan}{\strut$\_$this} \colorbox{LightCyan}{\strut$\_$process} \colorbox{LightCyan}{\strut ,} \colorbox{red!60!white}{\strut$\_$because} \colorbox{LightForestGreen}{\strut$\_$we} \colorbox{LightForestGreen}{\strut$\_$all} \colorbox{LightCyan}{\strut$\_$see} \colorbox{LightCyan}{\strut$\_$that} \colorbox{LightCyan}{\strut$\_$the} \colorbox{LightCyan}{\strut$\_$outcome} \colorbox{LightCyan}{\strut$\_$of} \colorbox{LightCyan}{\strut$\_$such} \colorbox{LightCyan}{\strut$\_$a} \colorbox{LightCyan}{\strut$\_$good} \colorbox{LightCyan}{\strut$\_$dialogue} \colorbox{LightCyan}{\strut$\_$will} \colorbox{LightCyan}{\strut$\_$be} \colorbox{LightCyan}{\strut$\_$positive} \colorbox{LightCyan}{\strut$\_$for} \colorbox{LightCyan}{\strut$\_$our} \colorbox{LightCyan}{\strut$\_$missions} \colorbox{LightCyan}{\strut.}

\end{description}